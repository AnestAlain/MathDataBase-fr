\documentclass[class=report,crop=false]{standalone}
\usepackage[screen]{../exo7book}

\begin{document}

%====================================================================
\chapitre{Dérivée d'une fonction}
%====================================================================

\insertvideo{5wpc0nsbBm4}{partie 1. Définition}

\insertvideo{TNfUA1PxosI}{partie 2. Calculs}

\insertvideo{t1uRmjrMnp8}{partie 3. Extremum local, théorème de Rolle}

\insertvideo{VdsiZNpZs2A}{partie 4. Théorème des accroissements finis}

\insertfiche{fic00013.pdf}{Fonctions dérivables}

%%%%%%%%%%%%%%%%%%%%%%%%%%%%%%%%%%%%%%%%%%%%%%%%%%%%%%%%%%%%%%%%
\section*{Motivation}

Nous souhaitons calculer $\sqrt{1,01}$ ou du moins en trouver une valeur approchée.
Comme $1,01$ est proche de $1$ et que $\sqrt{1}=1$ on se doute bien que $\sqrt{1,01}$
sera proche de $1$. Peut-on être plus précis ?
Si l'on appelle $f$ la fonction définie par $f(x)=\sqrt{x}$, alors la fonction $f$ est une fonction
continue en $x_0=1$. La continuité nous affirme que pour $x$ suffisamment proche de $x_0$,
$f(x)$ est proche de $f(x_0)$. Cela revient à dire que pour $x$ au voisinage de $x_0$ on approche $f(x)$
par la constante $f(x_0)$.

\myfigure{1.5}{
\tikzinput{fig_derive01}
}

Nous pouvons faire mieux qu'approcher notre fonction par une droite horizontale ! Essayons avec une droite
quelconque. Quelle droite se rapproche le plus du graphe de $f$ autour de $x_0$ ? Elle
doit passer par le point $(x_0,f(x_0))$ et doit \og coller \fg{}
le plus possible au graphe : c'est la tangente
au graphe en $x_0$.
Une équation de la tangente est
$$y = (x-x_0) f'(x_0) + f(x_0)$$
où $f'(x_0)$ désigne le nombre dérivé de $f$ en $x_0$.


On sait que pour $f(x)=\sqrt x$, on a $f'(x)=\frac{1}{2\sqrt x}$.
Une équation de la tangente en $x_0=1$ est donc $y=(x-1)\frac12+1$.
Et donc pour $x$ proche de $1$ on a $f(x) \approx (x-1)\frac12+1$.
Qu'est-ce que cela donne pour notre calcul de $\sqrt{1,01}$ ?
On pose $x=1,01$ donc $f(x) \approx 1+\frac12(x-1) = 1 + \frac{0,01}{2}=1,005$.
Et c'est effectivement une très bonne de approximation de $\sqrt{0,01}=1,00498\ldots$.
En posant $h=x-1$ on peut reformuler notre approximation en :
$\sqrt{1+h} \approx 1+\frac12h$ qui est valable pour $h$ proche de $0$.


Dans ce chapitre nous allons donc définir ce qu'est la dérivée d'une fonction et établir les formules des dérivées des fonctions usuelles.
Enfin, pour connaître l'erreur des approximations, il nous faudra travailler beaucoup plus
afin d'obtenir le théorème des accroissements finis.


%%%%%%%%%%%%%%%%%%%%%%%%%%%%%%%%%%%%%%%%%%%%%%%%%%%%%%%%%%%%%%%%
\section{Dérivée}

%---------------------------------------------------------------
\subsection{Dérivée en un point}

Soit $I$ un intervalle ouvert de $\Rr$ et $f : I \to \Rr$ une fonction. Soit $x_0 \in I$.

\begin{definition}
$f$ est \defi{dérivable en $x_0$}\index{fonction!derivable@dérivable} si le \evidence{taux d'accroissement} $\frac{f(x)-f(x_0)}{x-x_0}$
a une limite finie lorsque $x$ tend vers $x_0$.
La limite s'appelle alors le \defi{nombre dérivé} de $f$ en $x_0$ et est noté $f'(x_0)$. Ainsi
\mybox{$\displaystyle f'(x_0)= \lim_{x\to x_0} \frac{f(x)-f(x_0)}{x-x_0}$}
\end{definition}


\begin{definition}
$f$ est \defi{dérivable sur $I$} si $f$ est dérivable en tout point $x_0 \in I$.
La fonction $x \mapsto f'(x)$ est la \defi{fonction dérivée}\index{fonction!derivee@dérivée}\index{derivee@dérivée} de $f$,
elle se note $f'$ ou $\frac{df}{dx}$.
\end{definition}


\begin{exemple}
La fonction définie par $f(x)=x^2$  est dérivable en tout point $x_0 \in \Rr$. En effet :
$$\frac{f(x)-f(x_0)}{x-x_0} = \frac{x^2-x_0^2}{x-x_0} = \frac{(x-x_0)(x+x_0)}{x-x_0}=x+x_0 \xrightarrow[x \to x_0]{} 2x_0.$$
On a même montré que le nombre dérivé de $f$ en $x_0$ est $2x_0$, autrement dit : $f'(x)=2x$.
\end{exemple}

\begin{exemple}
Montrons que la dérivée de $f(x)=\sin x$ est $f'(x)=\cos x$.
Nous allons utiliser les deux assertions suivantes :
$$\frac{\sin x}{x} \xrightarrow[x\to 0]{} 1 \qquad \text{et} \qquad \sin p-\sin q = 2\sin \frac{p-q}{2}\cdot\cos\frac{p+q}{2}.$$

Remarquons déjà que la première assertion prouve $\frac{f(x)-f(0)}{x-0} = \frac{\sin x}{x} \to 1$ et donc $f$ est dérivable en $x_0=0$
et $f'(0)=1$.

Pour $x_0$ quelconque on écrit :
$$\frac{f(x)-f(x_0)}{x-x_0} = \frac{\sin x - \sin x_0}{x-x_0} = \frac{\sin \frac{x-x_0}{2}}{\frac{x-x_0}{2}} \cdot \cos \frac{x+x_0}{2}.$$
Lorsque $x\to x_0$ alors d'une part $\cos \frac{x+x_0}{2}\to \cos x_0$ et d'autre part
en posant $u=\frac{x-x_0}{2}$ alors $u\to 0$ et on a $\frac {\sin u}u \to 1$.
Ainsi $\frac{f(x)-f(x_0)}{x-x_0} \to \cos x_0$ et donc
$f'(x)=\cos x$.
\end{exemple}



%---------------------------------------------------------------
\subsection{Tangente}

La droite qui passe par les points distincts $(x_0,f(x_0))$ et $(x,f(x))$
a pour coefficient directeur $\frac{f(x)-f(x_0)}{x-x_0}$.
\`A la limite on trouve que le coefficient directeur de la tangente est $f'(x_0)$.
Une équation de la \defi{tangente}\index{tangente} au point $(x_0,f(x_0))$ est donc :
\mybox{$y = (x-x_0) f'(x_0) + f(x_0)$}

\myfigure{2}{
\tikzinput{fig_derive02}
}

%---------------------------------------------------------------
\subsection{Autres écritures de la dérivée}

Voici deux autres formulations de la dérivabilité de $f$ en $x_0$.

\begin{proposition}
\label{prop:ecrideriv}
\sauteligne
\begin{itemize}
  \item $f$ est dérivable en $x_0$ si et seulement si $\displaystyle \lim_{h\to 0} \frac{f(x_0+h)-f(x_0)}{h}$ existe et est finie.
  \item $f$ est dérivable en $x_0$ si et seulement s'il existe $\ell \in \Rr$ (qui sera $f'(x_0)$)
et une fonction $\epsilon : I \to \Rr$ telle que $\epsilon(x) \xrightarrow[x\to x_0]{} 0$ avec
$$f(x)=f(x_0)+(x-x_0) \ell + (x-x_0) \epsilon(x).$$
\end{itemize}
\end{proposition}

\begin{proof}
Il s'agit juste de reformuler la définition de $f'(x_0)$.
Par exemple, après division par $x-x_0$, la deuxième écriture devient
$$\frac{f(x)-f(x_0)}{x-x_0} = \ell + \epsilon(x).$$
\end{proof}

\begin{proposition}
Soit $I$ un intervalle ouvert, $x_0 \in I$ et soit $f : I \to \Rr$ une fonction.
\begin{itemize}
  \item Si $f$ est dérivable en $x_0$ alors $f$ est continue en $x_0$.
  \item Si $f$ est dérivable sur $I$ alors $f$ est continue sur $I$.
\end{itemize}
\end{proposition}

\begin{proof}
Supposons $f$ dérivable en $x_0$ et montrons qu'elle est aussi continue en ce point.
Voici une démonstration concise : partant de l'écriture alternative donnée dans la proposition \ref{prop:ecrideriv},
nous écrivons
$$f(x)= f(x_0)+\underbrace{(x-x_0) \ell}_{\to 0} + \underbrace{(x-x_0) \epsilon(x)}_{\to 0}.$$
Donc $f(x) \to f(x_0)$ lorsque $x \to x_0$ et ainsi $f$ est continue en $x_0$.

\medskip

On reprend cette démonstration sans utiliser les limites mais uniquement la définition
de continuité et dérivabilité :
fixons $\epsilon'>0$ et écrivons $f(x)= f(x_0)+(x-x_0) \ell + (x-x_0) \epsilon(x)$
grâce à la proposition \ref{prop:ecrideriv}, où $\epsilon(x) \xrightarrow[x\to x_0]{} 0$ et $\ell=f'(x_0)$.
Choisissons $\delta > 0$ de sorte qu'il vérifie tous les points suivants :
\begin{itemize}
  \item $\delta  \le 1$,
  \item $\delta |\ell| < \epsilon'$,
  \item si $|x-x_0|< \delta$ alors $|\epsilon(x)| < \epsilon'$ (c'est possible car $\epsilon(x) \to 0$).
\end{itemize}

Alors l'égalité ci-dessus devient :
\begin{align*}
\big|f(x)-f(x_0)\big|
  &= \big| (x-x_0)\ell + (x-x_0)\epsilon(x) \big| \\
  &\le |x-x_0|\cdot |\ell| + |x-x_0| \cdot |\epsilon(x)| \\
  &\le  \delta |\ell| \quad + \quad \delta \epsilon' \qquad \text{pour } |x-x_0| < \delta \\
  &\le  \epsilon' + \epsilon' = 2\epsilon' \\
\end{align*}

Nous venons de prouver que si $|x-x_0| < \delta$ alors $\big|f(x)-f(x_0)\big| < 2\epsilon'$,
ce qui exprime exactement que $f$ est continue en $x_0$.
\end{proof}

\begin{remarque*}
La réciproque est \textbf{fausse} : par exemple, la fonction valeur absolue
est continue en $0$ mais n'est pas dérivable en $0$.

\myfigure{1}{
\tikzinput{fig_derive03}
}

En effet, le taux d'accroissement de $f(x)=|x|$ en $x_0=0$ vérifie :
$$\frac{f(x)-f(0)}{x-0} = \frac{|x|}{x}=
\begin{cases}
+1 & \text{ si } x>0 \\
-1 & \text{ si } x < 0 \\
\end{cases}.
$$
Il y a bien une limite à droite (qui vaut $+1$), une limite à gauche (qui vaut $-1$) mais elles ne sont pas égales :
il n'y a pas de limite en $0$. Ainsi $f$ n'est pas dérivable en $x=0$.

Cela se lit aussi sur le dessin, il y a une demi-tangente à droite, une demi-tangente à gauche,
mais elles ont des directions différentes.
\end{remarque*}


%---------------------------------------------------------------
%\subsection{Mini-exercices}

\begin{miniexercices}
\sauteligne
\begin{enumerate}
  \item Montrer que la fonction $f(x)=x^3$ est dérivable en tout point $x_0\in \Rr$ et que $f'(x_0)=3x_0^2$.
  \item Montrer que la fonction $f(x)=\sqrt x$ est dérivable en tout point $x_0 >0$ et que $f'(x_0)=\frac{1}{2\sqrt{x_0}}$.
  \item Montrer que la fonction $f(x)=\sqrt x$ (qui est continue en $x_0=0$) n'est pas dérivable en $x_0=0$.
  \item Calculer l'équation de la tangente $(T_0)$ à la courbe d'équation $y=x^3-x^2-x$ au point d'abscisse $x_0=2$.
Calculer $x_1$ afin que la tangente $(T_1)$ au point d'abscisse $x_1$ soit parallèle à $(T_0)$.
  \item Montrer que si une fonction $f$ est paire et dérivable, alors $f'$ est une fonction impaire.
\end{enumerate}
\end{miniexercices}


%%%%%%%%%%%%%%%%%%%%%%%%%%%%%%%%%%%%%%%%%%%%%%%%%%%%%%%%%%%%%%%%
\section{Calcul des dérivées}

%---------------------------------------------------------------
\subsection{Somme, produit,...}


\begin{proposition}
Soient $f,g : I \to \Rr$ deux fonctions dérivables sur $I$. Alors pour tout $x \in I$ :
\begin{itemize}
  \item $(f+g)'(x) = f'(x)+g'(x)$
  \item $(\lambda f)'(x) = \lambda f'(x)$ \  où $\lambda$ est un réel fixé
  \item $(f \times g)'(x) = f'(x)g(x)+f(x)g'(x)$
  \item $\left(\frac{1}{f}\right)'(x)=-\frac{f'(x)}{f(x)^2}$ \   (si $f(x) \neq 0$)
  \item $\displaystyle\left(\frac{f}{g}\right)'(x)=\frac{f'(x)g(x)-f(x)g'(x)}{g(x)^2}$ \  (si $g(x) \neq 0$)
\end{itemize}
\end{proposition}

\begin{remarque*}
Il est plus facile de mémoriser les égalités de fonctions :
\mybox{$(f+g)'=f'+g' \qquad  (\lambda f)' = \lambda f' \qquad (f \times g)' = f'g+fg'$}
\mybox{$\displaystyle\left(\frac{1}{f}\right)'=-\frac{f'}{f^2} \qquad
\left(\frac{f}{g}\right)'=\frac{f'g-fg'}{g^2}$}
\end{remarque*}

\begin{proof}
Prouvons par exemple $(f \times g)' = f'g+fg'$.

Fixons $x_0 \in I$. Nous allons réécrire le taux d'accroissement de $f(x)\times g(x)$:
\begin{align*}
\frac{f(x)g(x)-f(x_0)g(x_0)}{x-x_0}
&=\frac{f(x)-f(x_0)}{x-x_0} g(x)+\frac{g(x)-g(x_0)}{x-x_0}f(x_0)\\
&\xrightarrow[x\to x_0]{} f'(x_0)g(x_0)+g'(x_0)f(x_0).
\end{align*}

%
% Fixons $x_0 \in I$. Comme $f$ est dérivable en $x_0$ alors
% on peut écrire  $f(x)= f(x_0)+(x-x_0) f'(x_0) + (x-x_0) \epsilon_1(x)$
% (voir la proposition \ref{prop:ecrideriv}) où $\epsilon_1(x) \to 0$ lorsque $x\to x_0$.
% De même $g$ est aussi dérivable en $x_0$ donc :
% $g(x)= g(x_0)+(x-x_0) g'(x_0) + (x-x_0) \epsilon_2(x)$
% où $\epsilon_2(x) \to 0$.
%
% Alors en développant on obtient :
% \begin{equation}
% \label{eq:derivprod}
% f(x)\times g(x) = f(x_0)\times g(x_0) + (x-x_0) \big( f'(x_0)g(x_0)+f(x_0)g'(x_0) \big) + (x-x_0) \epsilon_3(x)
% \end{equation}
% avec
% $$\epsilon_3(x)= f(x_0) \epsilon_2(x)+g(x_0)\epsilon_1(x)+(x-x_0) \big(
% f'(x_0)\epsilon_2(x) + g'(x_0) \epsilon_1(x) \big)$$
% qui vérifie bien $\epsilon_3(x) \to 0$ lorsque $x\to x_0$.
%
% Nous appliquons la proposition \ref{prop:ecrideriv} (cette fois dans le sens réciproque) :
% l'équation (\ref{eq:derivprod}) implique que la fonction $x \mapsto f(x)\times g(x)$ est dérivable en $x_0$,
% de dérivée $f'(x_0)g(x_0)+f(x_0)g'(x_0)$.

Ceci étant vrai pour tout $x_0 \in I$ la fonction $f\times g$ est dérivable sur $I$ de dérivée $f'g+fg'$.
\end{proof}

%---------------------------------------------------------------
\subsection{Dérivée de fonctions usuelles}

Le tableau de gauche est un résumé des principales formules à connaître, $x$ est une variable.
Le tableau de droite est celui des compositions (voir paragraphe suivant), $u$ représente une fonction $x\mapsto u(x)$.

\index{fonction!derivee@dérivée}
\begin{center}
%\noindent
\setlength{\arrayrulewidth}{0.05mm}
%\begin{tabular}{|l|l|l|} \hline
\begin{tabular}[t]{|c|c@{\vrule depth 1.2ex height 3ex width 0mm \ }|}
\hline
\textbf{Fonction}         & \textbf{Dérivée} \\ \hline
   $x^n$         & $nx^{n-1}$  \quad ($n \in \Zz$)   \\ \hline
   $\frac 1x$    & $-\frac{1}{x^2}$              \\ \hline
   $\sqrt{x}$    & $\frac12 \frac1{\sqrt{x}}$   \\ \hline
   $x^\alpha$   & $\alpha x^{\alpha-1}$  \quad ($\alpha\in\Rr$)  \\ \hline
   $e^x$         & $e^x$                        \\ \hline
   $\ln x$       & $\frac 1x$                   \\ \hline
   $\cos x$      & $-\sin x$                    \\ \hline
   $\sin x$      & $\cos x$                     \\ \hline
   $\tan x$      & $1+\tan^2 x = \frac{1}{\cos^2 x}$        \\ \hline
\end{tabular}
\hspace*{2cm}
\begin{tabular}[t]{|c|c@{\vrule depth 1.2ex height 3ex width 0mm \ }|}
\hline
\textbf{Fonction}         & \textbf{Dérivée} \\ \hline
   $u^n$         & $nu'u^{n-1}$  \quad  ($n \in \Zz$)   \\ \hline
   $\frac 1u$    & $-\frac{u'}{u^2}$              \\ \hline
   $\sqrt{u}$    & $\frac12 \frac{u'}{\sqrt{u}}$   \\ \hline
   $u^\alpha$   & $\alpha u' u^{\alpha-1}$ \quad ($\alpha\in\Rr$)  \\ \hline
   $e^u$         & $u'e^u$                        \\ \hline
   $\ln u$       & $\frac {u'}{u}$                   \\ \hline
   $\cos u$      & $-u'\sin u$                    \\ \hline
   $\sin u$      & $u'\cos u$                     \\ \hline
   $\tan u$      & $u'(1+\tan^2 u) = \frac{u'}{\cos^2 u}$        \\ \hline
\end{tabular}
\hfill
\end{center}

\begin{remarque*}
\sauteligne
\begin{itemize}
  \item Notez que les formules pour $x^n$, $\frac 1x$, $\sqrt x$ et $x^\alpha$
sont aussi des conséquences de la dérivée de l'exponentielle.
Par exemple $x^\alpha = e^{\alpha \ln x}$ et donc
$$\frac{d }{dx}(x^\alpha) = \frac{d}{dx} (e^{\alpha \ln x}) = \alpha \frac{1}{x} e^{\alpha \ln x} =\alpha \frac 1x x^{\alpha}=\alpha x^{\alpha-1}.$$

  \item Si vous devez dériver une fonction avec un exposant dépendant de $x$ il faut absolument
repasser à la forme exponentielle.
Par exemple si $f(x)= 2^x$ alors on réécrit d'abord $f(x)=e^{x\ln 2}$ pour pouvoir calculer
$f'(x)=\ln 2 \cdot  e^{x\ln 2} = \ln 2  \cdot 2^x$.
\end{itemize}

\end{remarque*}




%---------------------------------------------------------------
\subsection{Composition}

\begin{proposition}
Si $f$ est dérivable en $x$ et $g$ est dérivable en $f(x)$ alors $g\circ f$ est
dérivable en $x$ de dérivée :
\mybox{$\big( g \circ f \big)'(x) = g'\big( f(x) \big) \cdot f'(x)$}
\end{proposition}

\begin{proof}
Faisons l'hypothèse que $f(x) \neq f(x_0)$ pour $x$ proche de $x_0$ (avec $x \neq x_0$). 
La preuve est alors similaire à celle ci-dessus pour le produit en écrivant cette fois :
\begin{align*}
\frac{g\circ f(x)-g\circ f(x_0)}{x-x_0} 
&= \frac{g\big( f(x)\big)-g\big( f(x_0)\big)}{f(x)-f(x_0)} \times \frac{f(x)-f(x_0)}{x-x_0} \\
&\xrightarrow[x\to x_0]{} g'\big(f(x_0)\big) \times f'(x_0).
\end{align*}
\end{proof}


\begin{exemple}
Calculons la dérivée de $\ln(1+x^2)$. Nous avons $g(x)=\ln(x)$ avec $g'(x) = \frac 1x$ ; et
 $f(x)=1+x^2$ avec $f'(x) = 2x$.
Alors la dérivée de $\ln(1+x^2)=g\circ f(x)$ est
$$\big( g \circ f \big)'(x) = g'\big( f(x) \big) \cdot f'(x) = g'\big( 1+x^2 \big) \cdot 2x = \frac{2x}{1+x^2}.$$
\end{exemple}


\begin{corollaire}
Soit $I$ un intervalle ouvert. Soit $f : I \to J$ dérivable et bijective dont on note
$f^{-1} : J \to I$ la bijection réciproque. Si $f'$ ne s'annule pas sur $I$ alors $f^{-1}$ est dérivable
et on a pour tout $x \in J$ :
\mybox{$\displaystyle \big(f^{-1}\big)'(x)= \frac{1}{f'\big( f^{-1}(x) \big)} $}
\end{corollaire}


\begin{proof}
Notons $g=f^{-1}$ la bijection réciproque de $f$.
Soit $y_0 \in J$ et $x_0\in I$ tel que $y_0=f(x_0)$. Le taux d'accroissement de $g$ en $y_0$ est~:
$$\frac{g(y)-g(y_0)}{y-y_0}=\frac{g(y)-x_0}{f\big(g(y)\big) - f(x_0)}$$
Lorsque $y\to y_0$ alors $g(y)\to g(y_0) = x_0$ et donc ce taux d'accroissement tend vers $\frac{1}{f'(x_0)}$.
Ainsi $g'(y_0)=\frac{1}{f'(x_0)}$.
\end{proof}

\begin{remarque*}
Il peut être plus simple de retrouver la formule à chaque fois
en dérivant l'égalité
$$f\big( g(x) \big)  = x$$
où $g=f^{-1}$ est la bijection réciproque de $f$.

En effet à droite la dérivée de $x$ est $1$ ;
à gauche la dérivée de $f\big( g(x) \big) = f \circ g(x)$ est $f'\big(g(x)\big) \cdot g'(x)$.
L'égalité $f\big( g(x) \big)  = x$ conduit donc à l'égalité des dérivées :
$$f'\big(g(x)\big)\cdot g'(x) = 1.$$
Mais $g=f^{-1}$ donc
$$\big(f^{-1}\big)'(x)= \frac{1}{f'\big( f^{-1}(x) \big)}.$$
\end{remarque*}


\begin{exemple}
Soit $f : \Rr \to \Rr$ la fonction définie par $f(x)=x+\exp(x)$.
\'Etudions $f$ en détail.

Tout d'abord :
\begin{enumerate}
  \item $f$ est dérivable car $f$ est la somme de deux fonctions dérivables. En particulier $f$ est continue.
  \item $f$ est strictement croissante car $f$ est la somme de deux fonctions strictement croissante.
  \item $f$ est une bijection car $\lim_{x\to-\infty} f(x)=-\infty$ et $\lim_{x\to+\infty} f(x)=+\infty$.
  \item $f'(x) = 1 + \exp(x)$ ne s'annule jamais (pour tout $x\in \Rr$).
\end{enumerate}

\bigskip

Notons $g = f^{-1}$ la bijection réciproque de $f$. Même si on ne sait pas a priori exprimer $g$, on peut malgré tout connaître des informations sur
cette fonction : par le corollaire ci-dessus $g$ est dérivable
et l'on calcule $g'$ en dérivant l'égalité $f\big( g(x) \big)  = x$. Ce qui donne
$f'\big(g(x)\big) \cdot g'(x) = 1$
et donc ici
$$g'(x) = \frac{1}{f'\big( g(x) \big)} = \frac{1}{1+\exp\big( g(x) \big)}.$$
Pour cette fonction $f$ particulière on peut préciser davantage :
comme  $f\big( g(x)\big) = x$ alors $g(x)+\exp\big( g(x) \big)=x$ donc
$\exp\big( g(x) \big)=x-g(x)$. Cela conduit à :
$$g'(x) =  \frac{1}{1+x -g(x)}.$$

\bigskip

\myfigure{1}{
\tikzinput{fig_derive13}
}

Par exemple  $f(0)=1$ donc $g(1)=0$ et donc $g'(1)=\frac12$.
Autrement dit $\big(f^{-1}\big)'(1)= \frac{1}{2}$. L'équation de la tangente
au graphe de $f^{-1}$ au point d'abscisse $x_0=1$ est donc $y=\frac12 (x-1)$.
\end{exemple}


%---------------------------------------------------------------
\subsection{Dérivées successives}

Soit $f : I \to \Rr$ une fonction dérivable et soit $f'$ sa dérivée.
Si la fonction $f' : I \to \Rr$ est aussi dérivable on note
$f''=(f')'$ la \defi{dérivée seconde}\index{derivee seconde@dérivée seconde} de $f$.
Plus généralement on note :
$$f^{(0)} = f, \quad f^{(1)} = f', \quad f^{(2)} = f'' \quad \text{ et } \quad f^{(n+1)} = \big(f^{(n)}\big)'$$

Si la \defi{dérivée $n$-ième} $f^{(n)}$ existe  on dit que $f$ est \defi{$n$ fois dérivable}.


\begin{theoreme}[Formule de Leibniz]
\index{formule!de Leibniz}
\sauteligne
\mybox{$\displaystyle \big( f \cdot g \big)^{(n)} =  f^{(n)} \cdot g + \binom{n}{1}\ f^{(n-1)}\cdot g^{(1)}
+ \cdots + \binom{n}{k} \ f^{(n-k)} \cdot g^{(k)}+\cdots + f \cdot g^{(n)}$}
\end{theoreme}

Autrement dit :
$$\big( f \cdot g \big)^{(n)} = \sum_{k=0}^n \binom{n}{k} \ f^{(n-k)} \cdot g^{(k)}.$$

La démonstration est similaire à celle de la formule du binôme de Newton
et les coefficients que l'on obtient sont les mêmes.
\begin{exemple}
\sauteligne
\begin{itemize}
  \item Pour $n=1$ on retrouve $(f\cdot g)'= f' g + f g'$.
  \item Pour $n=2$, on a $(f\cdot g)''= f''g + 2f' g' + fg''$.
\end{itemize}
\end{exemple}


\begin{exemple}
Calculons les dérivées $n$-ème de $\exp(x) \cdot (x^2+1)$ pour tout $n \ge 0$.
Notons $f(x)=\exp(x)$ alors $f'(x)=\exp(x)$, $f''(x)=\exp(x)$,..., $f^{(k)}(x)=\exp(x)$.
Notons $g(x)=x^2+1$ alors $g'(x)=2x$, $g''(x)=2$ et pour $k\ge 3$, $g^{(k)}(x)=0$.


Appliquons la formule de Leibniz :
\begin{multline*}
\big( f \cdot g \big)^{(n)}(x) =  f^{(n)}(x) \cdot g(x) + \binom{n}{1}\ f^{(n-1)}(x)\cdot g^{(1)}(x)
+\\
+\binom{n}{2}\  f^{(n-2)}(x)\cdot g^{(2)}(x) + \binom{n}{3}\ f^{(n-3)}(x)\cdot g^{(3)}(x) + \cdots
\end{multline*}

On remplace $f^{(k)}(x)=\exp(x)$ et on sait que $g^{(3)}(x)=0$, $g^{(4)}(x)=0$,\ldots
Donc cette somme ne contient que les trois premiers termes :
$$\big( f \cdot g \big)^{(n)}(x) =  \exp(x) \cdot (x^2+1) + \binom{n}{1}\ \exp(x) \cdot 2x
+ \binom{n}{2}\  \exp(x) \cdot 2.$$
Que l'on peut aussi écrire :
$$\big( f \cdot g \big)^{(n)}(x) =  \exp(x) \cdot \Big(x^2 + 2nx + n(n-1)+1  \Big).$$

\end{exemple}


%---------------------------------------------------------------
%\subsection{Mini-exercices}

\begin{miniexercices}
\sauteligne
\begin{enumerate}
  \item Calculer les dérivées des fonctions suivantes :
$f_1(x) = x\ln x$, $f_2(x)=\sin \frac 1x$, $f_3(x)=\sqrt{1+\sqrt{1+x^2}}$, $f_4(x)= \big(\ln(\frac{1+x}{1-x})\big)^{\frac13}$,
$f_5(x) = x^x$, $f_6(x) = \arctan x + \arctan \frac 1x$.
  \item On note $\Delta(f)=\frac{f'}{f}$. Calculer $\Delta(f\times g)$.
  \item Soit $f : ]1, +\infty[ \to ]-1, +\infty[$ définie par $f(x)= x\ln (x) - x$.
Montrer que $f$ est une bijection. Notons $g=f^{-1}$. Calculer $g(0)$ et $g'(0)$.
  \item Calculer les dérivées successives de $f(x)=\ln(1+x)$.
  \item Calculer les dérivées successives de $f(x)=\ln(x) \cdot x^3$.
\end{enumerate}
\end{miniexercices}


%%%%%%%%%%%%%%%%%%%%%%%%%%%%%%%%%%%%%%%%%%%%%%%%%%%%%%%%%%%%%%%%
\section{Extremum local, théorème de Rolle}

%---------------------------------------------------------------
\subsection{Extremum local}


Soit $f : I \to \Rr$ une fonction définie sur un intervalle $I$.
\begin{definition}
\sauteligne
\begin{itemize}
  \item On dit que $x_0$ est un \defi{point critique}\index{point critique} de $f$ si $f'(x_0)=0$.

  \item On dit que $f$ admet un \defi{maximum local en $x_0$}\index{maximum} (resp. un \defi{minimum local
en $x_0$}\index{minimum}) s'il existe un intervalle ouvert $J$ contenant $x_0$  tel que
$$\text{pour tout } x\in I \cap J \quad f(x) \le f(x_0)$$
(resp. $f(x) \ge f(x_0)$).
  \item On dit que $f$ admet un \defi{extremum local en $x_0$}\index{extremum} si $f$ admet un maximum
local ou un minimum local en ce point.
\end{itemize}
\end{definition}

\myfigure{1.5}{
\tikzinput{fig_derive04}
}

Dire que $f$ a un maximum local en $x_0$ signifie que $f(x_0)$ est la plus grande des valeurs $f(x)$ pour les $x$ proches de $x_0$.
On dit que $f : I \to \Rr$ admet un \defi{maximum global} en $x_0$ si pour toutes les autres valeurs $f(x)$, $x\in I$,
on a $f(x) \le f(x_0)$ (on ne regarde donc pas seulement les $f(x)$ pour $x$ proche de $x_0$).
Bien sûr un maximum global est aussi un maximum local, mais la réciproque est fausse.

\begin{theoreme}
\label{th:extremum}
Soit $I$ un intervalle ouvert et $f : I \to \Rr$ une fonction
dérivable. Si $f$ admet un maximum local (ou un minimum local)
en $x_0$ alors $f'(x_0)=0$.
\end{theoreme}

En d'autres termes, un maximum local (ou un minimum local) $x_0$ est toujours un point critique.
Géométriquement, au point $(x_0,f(x_0))$
la tangente au graphe est horizontale.

\myfigure{1.5}{
\tikzinput{fig_derive05}
}

\begin{exemple}
\'Etudions les extremums de la fonction $f_\lambda$ définie par
$f_\lambda(x)= x^3+\lambda x$ en fonction du paramètre $\lambda \in \Rr$.
La dérivée est $f_\lambda'(x) = 3x^2+\lambda$. Si $x_0$ est un extremum local
alors $f'_\lambda(x_0)=0$.

\begin{itemize}
  \item Si $\lambda>0$ alors $f'_\lambda(x)>0$ et ne s'annule jamais il n'y a pas de points critiques
donc pas non plus d'extremums. En anticipant sur la suite : $f_\lambda$ est strictement croissante sur $\Rr$.

  \item Si $\lambda = 0$ alors $f_\lambda'(x)=3x^2$. Le seul point critique est $x_0=0$.
Mais ce n'est ni un maximum local, ni un minimum local. En effet si $x<0$, $f_0(x)<0=f_0(0)$
et si $x>0$, $f_0(x)>0=f_0(0)$.

  \item Si $\lambda <0$ alors $f'_\lambda(x)= 3x^2-|\lambda|
= 3\big(x+\sqrt{\frac{|\lambda|}{3}}\big)\big(x-\sqrt{\frac{|\lambda|}{3}}\big)$.
Il y a deux points critiques $x_1= -\sqrt{\frac{|\lambda|}{3}}$ et $x_2=+\sqrt{\frac{|\lambda|}{3}}$.
En anticipant sur la suite : $f_\lambda'(x) > 0$ sur $]-\infty,x_1[$ et $]x_2,+\infty[$
et $f_\lambda'(x) < 0$ sur $]x_1,x_2[$ ;
maintenant $f_\lambda$ est croissante sur $]-\infty,x_1[$, puis décroissante sur $]x_1,x_2[$,
donc $x_1$ est un maximum local. D'autre part $f_\lambda$ est décroissante sur $]x_1,x_2[$
puis croissante sur $]x_2,+\infty[$ donc $x_2$ est un minimum local.

\medskip

\myfigure{1.3}{
\tikzinput{fig_derive12a}
\quad
\tikzinput{fig_derive12b}
\quad
\tikzinput{fig_derive12c}
}

\end{itemize}
\end{exemple}

\begin{remarque*}
\sauteligne
\begin{enumerate}
  \item La réciproque du théorème \ref{th:extremum}  est fausse.
Par exemple la fonction $f : \Rr \to \Rr$, définie par $f(x)= x^3$
vérifie $f'(0)=0$ mais $x_0=0$ n'est ni maximum local ni un minimum local.

  \item L'intervalle du théorème \ref{th:extremum} est ouvert. Pour le cas d'un intervalle fermé, il faut faire attention
aux extrémités. Par exemple si $f : [a,b] \to \Rr$ est une fonction dérivable qui admet un extremum en $x_0$,
alors on est dans l'une des situations suivantes :
\begin{itemize}
  \item $x_0= a$,
  \item $x_0 =b$,
  \item $x_0 \in ]a,b[$ et dans ce cas on a bien $f'(x_0)=0$ par le théorème \ref{th:extremum}.
\end{itemize}
Aux extrémités on ne peut rien dire pour $f'(a)$ et $f'(b)$, comme le montre les différents maximums sur
les dessins suivants.

\myfigure{0.9}{
\tikzinput{fig_derive06} \ 
\tikzinput{fig_derive07} \ 
\tikzinput{fig_derive08}
}

  \item Pour déterminer $\max_{[a,b]} f$ et $\min_{[a,b]} f$ (où $f:[a,b]\to \Rr$ est une fonction dérivable) il faut comparer
les valeurs de $f$ aux différents points critiques et en $a$ et en $b$.
\end{enumerate}
\end{remarque*}

\begin{proof}[Preuve du théorème]
Supposons que $x_0$ soit un maximum local de $f$, soit donc $J$ l'intervalle ouvert de la définition
contenant $x_0$ tel que pour tout $x\in I \cap J$ on a $f(x) \le f(x_0)$.
\begin{itemize}
  \item Pour $x \in I\cap J$ tel que $x < x_0$ on a $f(x)-f(x_0) \le 0$ et $x-x_0<0$ donc $\frac{f(x)-f(x_0)}{x-x_0} \ge 0$
et donc à la limite $\lim_{x \to x_0^-} \frac{f(x)-f(x_0)}{x-x_0} \ge 0$.
  \item Pour $x \in I\cap J$ tel que $x > x_0$ on a $f(x)-f(x_0) \le 0$ et $x-x_0>0$ donc $\frac{f(x)-f(x_0)}{x-x_0} \le 0$
et donc à la limite $\lim_{x \to x_0^+} \frac{f(x)-f(x_0)}{x-x_0} \le 0$.
\end{itemize}
Or $f$ est dérivable en $x_0$ donc
$$\lim_{x \to x_0^-} \frac{f(x)-f(x_0)}{x-x_0} = \lim_{x \to x_0^+} \frac{f(x)-f(x_0)}{x-x_0} = f'(x_0).$$
La première limite est positive, la seconde est négative, la seule possibilité est que $f'(x_0)=0$.
\end{proof}


%---------------------------------------------------------------
\subsection{Théorème de Rolle}

\begin{theoreme}[Théorème de Rolle]
\index{theoreme@théorème!de Rolle}
\label{th:rolle}
Soit $f:[a,b] \to \Rr$ telle que
\begin{itemize}
  \item $f$ est continue sur $[a,b]$,
  \item $f$ est dérivable sur $]a,b[$,
  \item $f(a)=f(b)$.
\end{itemize}
Alors il existe $c \in ]a,b[$  tel que $f'(c)=0$.
\end{theoreme}

\myfigure{1.5}{
\tikzinput{fig_derive09}
}

Interprétation géométrique : il existe au moins un point du graphe de $f$ où la tangente est horizontale.

\begin{proof}
Tout d'abord, si $f$ est constante sur $[a,b]$ alors n'importe quel $c\in]a,b[$ convient.
Sinon il existe $x_0 \in [a,b]$ tel que $f(x_0) \neq f(a)$.
Supposons par exemple $f(x_0) > f(a)$. Alors $f$ est continue sur l'intervalle fermé et borné
$[a,b]$, donc elle admet un maximum en un point $c\in[a,b]$.
Mais $f(c) \ge f(x_0) > f(a)$ donc $c \neq a$. De même comme $f(a)=f(b)$ alors  $c\neq b$.
Ainsi $c\in ]a,b[$.
En $c$, $f$ est donc dérivable et admet un maximum (local) donc $f'(c)=0$.
\end{proof}


\begin{exemple}
Soit $P(X) = (X-\alpha_1)(X-\alpha_2)\cdots(X-\alpha_n)$
un polynôme ayant $n$ racines réelles différentes : $\alpha_1< \alpha_2 < \cdots < \alpha_n$.

\begin{enumerate}
  \item \emph{Montrons que $P'$ a $n-1$ racines distinctes.}

On considère $P$ comme une fonction polynomiale $x \mapsto P(x)$.
$P$ est une fonction continue et dérivable sur $\Rr$.
Comme $P(\alpha_1)=0=P(\alpha_2)$ alors par le théorème de Rolle
il existe $c_1 \in ]\alpha_1,\alpha_2[$
tel que $P'(c_1)=0$.
Plus généralement, pour $1 \le k \le n-1$, comme $P(\alpha_k)=0=P(\alpha_{k+1})$ alors
le théorème de Rolle implique l'existence de $c_k \in ]\alpha_k,\alpha_{k+1}[$ tel que $P'(c_k)=0$.
Nous avons bien trouvé $n-1$ racines de $P'$ : $c_1< c_2 < \cdots < c_{n-1}$.
Comme $P'$ est un polynôme de degré $n-1$, toutes ses racines sont réelles et distinctes.

  \item \emph{Montrons que $P+P'$ a $n-1$ racines distinctes.}

L'astuce consiste à considérer la fonction auxiliaire $f(x)=P(x)\exp x$.
$f$ est une fonction continue et dérivable sur $\Rr$.
$f$ s'annule comme $P$ en $\alpha_1,\ldots,\alpha_n$.
La dérivée de $f$ est $f'(x)=\big(P(x)+P'(x)\big) \exp x$.
Donc par le théorème de Rolle, pour chaque  $1 \le k \le n-1$, comme $f(\alpha_k)=0=f(\alpha_{k+1})$ alors
il existe $\gamma_k \in ]\alpha_k,\alpha_{k+1}[$ tel que $f'(\gamma_k)=0$.
Mais comme la fonction exponentielle ne s'annule jamais alors
$(P+P')(\gamma_k)=0$.
Nous avons bien trouvé $n-1$ racines distinctes de $P+P'$ : $\gamma_1 < \gamma_2 < \cdots < \gamma_{n-1}$.



  \item \emph{Déduisons-en que $P+P'$ a toutes ses racines réelles.}

$P+P'$ est un polynôme à coefficients réels qui admet $n-1$ racines réelles.
Donc $(P+P')(X)=(X-\gamma_1)\cdots(X-\gamma_{n-1}) Q(X)$ où $Q(x)=X-\gamma_n$ est un polynôme de degré $1$.
Comme $P+P'$ est à coefficients réels et que les $\gamma_i$ sont aussi réels, ainsi $\gamma_n \in \Rr$.
Ainsi on a obtenu une $n$-ème racine réelle $\gamma_n$ (pas nécessairement distincte des autres $\gamma_i$).

\end{enumerate}

\end{exemple}

%---------------------------------------------------------------
%\subsection{Mini-exercices}

\begin{miniexercices}
\sauteligne
\begin{enumerate}
  \item Dessiner le graphe de fonctions vérifiant :  $f_1$ admet deux minimums locaux et un maximum local ;
$f_2$ admet un minimum local qui n'est pas global et un maximum local qui est global ;
$f_3$ admet une infinité d'extremums locaux ; $f_4$ n'admet aucun extremum local.
  \item Calculer en quel point la fonction $f(x)=ax^2+bx+c$ admet un extremum local.
  \item Soit $f : [0,2] \to \Rr$ une fonction deux fois dérivable telle que $f(0)=f(1)=f(2)=0$.
Montrer qu'il existe $c_1,c_2$ tels que $f'(c_1)=0$ et $f'(c_2)=0$. Montrer qu'il existe
$c_3$ tel que $f''(c_3)=0$.
  \item Montrer que chacune des trois hypothèses du théorème de Rolle est nécessaire.
\end{enumerate}
\end{miniexercices}


%%%%%%%%%%%%%%%%%%%%%%%%%%%%%%%%%%%%%%%%%%%%%%%%%%%%%%%%%%%%%%%%
\section{Théorème des accroissements finis}

%---------------------------------------------------------------
\subsection{Théorème des accroissements finis}


\begin{theoreme}[Théorème des accroissements finis]
\index{theoreme@théorème!des accroissements finis}
Soit $f:[a,b] \to \Rr$ une fonction continue sur $[a,b]$ et dérivable sur $]a,b[$.
Il existe $c\in]a,b[$ tel que
\mybox{$f(b)-f(a)= f'(c) \; (b-a)$}
\end{theoreme}

\myfigure{1.5}{
\tikzinput{fig_derive10}
}

Interprétation géométrique : il existe au moins un point du graphe de $f$ où la tangente est
parallèle à la droite $(AB)$ où $A=(a,f(a))$ et $B=(b,f(b))$.

\begin{proof}
Posons $\ell = \frac{f(b)-f(a)}{b-a}$ et $g(x) = f(x) - \ell \cdot (x-a)$.
Alors $g(a)=f(a)$, $g(b)=f(b)- \frac{f(b)-f(a)}{b-a} \cdot (b-a) = f(a)$.
Par le théorème de Rolle, il existe $c \in ]a,b[$ tel que $g'(c) =0$.
Or $g'(x) = f'(x) - \ell$.
Ce qui donne $f'(c)= \frac{f(b)-f(a)}{b-a}$.
\end{proof}


%---------------------------------------------------------------
\subsection{Fonction croissante et dérivée}

\begin{corollaire}
Soit $f:[a,b] \to \Rr$ une fonction continue sur $[a,b]$ et dérivable sur $]a,b[$.
\begin{enumerate}
  \item \label{it:crois}$\forall x \in ]a,b[ \quad f'(x) \ge 0 \quad \iff \quad$ $f$ est croissante ;
  \item $\forall x \in ]a,b[ \quad f'(x) \le 0 \quad \iff \quad$ $f$ est décroissante ;
  \item  $\forall x \in ]a,b[ \quad f'(x) = 0 \quad \iff \quad$ $f$ est constante ;
  \item \label{it:stcrois} $\forall x \in ]a,b[ \quad f'(x) > 0 \quad \implies \quad$ $f$ est strictement croissante ;
  \item \label{it:stdec} $\forall x \in ]a,b[ \quad f'(x) < 0 \quad \implies \quad$ $f$ est strictement décroissante.
\end{enumerate}
\end{corollaire}

\begin{remarque*}
La réciproque au point (\ref{it:stcrois}) (et aussi au  (\ref{it:stdec})) est fausse.
Par exemple la fonction $x \mapsto x^3$ est strictement croissante et pourtant sa dérivée s'annule en $0$.
\end{remarque*}


\begin{proof}
Prouvons par exemple (\ref{it:crois}).

Sens $\Longrightarrow$. \quad  Supposons d'abord la dérivée positive.
Soient $x,y \in ]a,b[$ avec $x\le y$. Alors
par le théorème des accroissements finis, il existe $c\in]x,y[$ tel que
$f(x)-f(y) = f'(c) (x-y)$. Mais $f'(c) \ge 0$ et $x-y \le 0$ donc $f(x)-f(y) \le 0$.
Cela implique que $f(x) \le f(y)$. Ceci étant vrai pour tout $x,y$ alors $f$ est croissante.

\medskip

Sens $\Longleftarrow$. \quad Réciproquement, supposons que $f$ est croissante. Fixons $x\in]a,b[$.
Pour tout $y > x$ nous avons $y-x>0$ et $f(y)-f(x)\ge 0$, ainsi
le taux d'accroissement vérifie $\frac{f(y)-f(x)}{y-x}\ge 0$. \`A la limite,
quand $y \to x$, ce taux d'accroissement
tend vers la dérivée de $f$ en $x$ et donc $f'(x) \ge 0$.
\end{proof}


%---------------------------------------------------------------
\subsection{Inégalité des accroissements finis}


\begin{corollaire}[Inégalité des accroissements finis]
\index{inegalite@inégalité!des accroissements finis}
Soit $f : I \to \Rr$ une fonction dérivable sur un intervalle $I$ ouvert.
S'il existe une constante $M$ telle que pour tout $x \in I$, $\big|f'(x)\big| \le M$ alors
\mybox{$\forall x,y \in I \qquad \big| f(x)-f(y) \big| \le M |x-y|$}
\end{corollaire}

\begin{proof}
Fixons $x,y \in I$, il existe alors $c\in]x,y[$ ou $]y,x[$ tel que $f(x)-f(y)=f'(c)(x-y)$ et
comme $|f'(c)| \le M$ alors $\big| f(x)-f(y) \big| \le M |x-y|$.
\end{proof}

\begin{exemple}
Soit $f(x)=\sin(x)$. Comme $f'(x)=\cos x$ alors $|f'(x)| \le 1$ pour tout $x\in \Rr$.
L'inégalité des accroissements finis s'écrit alors :
$$\text{pour tout } x,y \in \Rr \qquad |\sin x - \sin y | \le |x-y|.$$

En particulier si l'on fixe $y=0$ alors
on obtient
\mybox{$|\sin x| \le |x|$}
ce qui est particulièrement intéressant pour $x$ proche de $0$.

\myfigure{0.8}{
\tikzinput{fig_derive11}
}
\end{exemple}


%---------------------------------------------------------------
\subsection{Règle de l'Hospital}


\begin{corollaire}[Règle de l'Hospital]
\index{regle@règle!de l'Hospital}
Soient $f,g : I \to \Rr$ deux fonctions dérivables et soit $x_0\in I$.
On suppose que
\begin{itemize}
  \item $f(x_0)=g(x_0)=0$,
  \item $\forall x \in I\setminus\{x_0\} \quad g'(x)\neq0$.
\end{itemize}
\mybox{Si \quad $\displaystyle \lim_{x\to x_0} \frac{f'(x)}{g'(x)} = \ell \quad (\in \Rr)$
\quad alors \quad $\displaystyle \lim_{x\to x_0} \frac{f(x)}{g(x)} = \ell.$}
\end{corollaire}


\begin{proof}
Fixons $a\in I\setminus\{x_0\}$ avec par exemple $a<x_0$.
Soit $h : I \to \Rr$ définie par $h(x)=g(a)f(x)-f(a)g(x)$.
Alors
\begin{itemize}
  \item $h$ est continue sur $[a,x_0] \subset I$,
  \item $h$ est dérivable sur $]a,x_0[$,
  \item $h(x_0)=h(a)=0$.
\end{itemize}
Donc par le théorème de Rolle il existe $c_a \in ]a,x_0[$ tel que $h'(c_a)=0$.
Or $h'(x)=g(a)f'(x)-f(a)g'(x)$ donc $g(a)f'(c_a)-f(a)g'(c_a)=0$.
Comme $g'$ ne s'annule pas sur $I\setminus\{x_0\}$ cela conduit à
$\frac{f(a)}{g(a)}=\frac{f'(c_a)}{g'(c_a)}$.
Comme $a<c_a<x_0$ lorsque l'on fait tendre $a$ vers $x_0$ on obtient $c_a \to x_0$.
Cela implique
$$\lim_{a\to x_0} \frac{f(a)}{g(a)} = \lim_{a\to x_0} \frac{f'(c_a)}{g'(c_a)} = \lim_{c_a\to x_0} \frac{f'(c_a)}{g'(c_a)} = \ell.$$
\end{proof}

\begin{exemple}
Calculer la limite en $1$ de $\frac{\ln(x^2+x-1)}{\ln(x)}$.
On vérifie que :
\begin{itemize}
  \item $f(x)=\ln(x^2+x-1)$, $f(1)=0$, $f'(x)=\frac{2x+1}{x^2+x-1}$,
  \item $g(x)=\ln(x)$, $g(1)=0$, $g'(x)=\frac 1x$,
  \item Prenons $I=]0,1]$, $x_0=1$, alors $g'$ ne s'annule pas sur $I\setminus\{x_0\}$.
\end{itemize}


$$\frac{f'(x)}{g'(x)} = \frac{2x+1}{x^2+x-1} \times x = \frac{2x^2+x}{x^2+x-1} \xrightarrow[x\to 1]{} 3.$$
Donc
$$\frac{f(x)}{g(x)} \xrightarrow[x\to 1]{} 3.$$
\end{exemple}

%---------------------------------------------------------------
%\subsection{Mini-exercices}

\begin{miniexercices}
\sauteligne
\begin{enumerate}
  \item Soit $f(x) = \frac{x^3}{3}+\frac{x^2}{2}-2x+2$. \'Etudier la fonction $f$. Tracer son graphe.
Montrer que $f$ admet un minimum local et un maximum local.

   \item Soit $f(x)=\sqrt{x}$. Appliquer le théorème des accroissements finis sur l'intervalle $[100,101]$.
En déduire l'encadrement $10+\frac{1}{22} \le \sqrt{101} \le 10 + \frac{1}{20}$.

  \item Appliquer le théorème des accroissements finis pour montrer que $\ln (1+x)-\ln(x) < \frac 1x$ (pour tout $x>0$).

  \item Soit $f(x) = e^x$. Que donne l'inégalité des accroissements finis sur $[0,x]$ ?
%En déduire que pour tout $x \ge 0$, $e^x-1 \le xe^x$.

  \item Appliquer la règle de l'Hospital pour calculer les limites suivantes (quand $x\to 0$):
$\displaystyle \frac{x}{(1+x)^n-1}$ ; $\displaystyle \frac{\ln(x+1)}{\sqrt x}$ ;
$\displaystyle \frac{1-\cos x}{\tan x}$ ; $\displaystyle \frac{x-\sin x}{x^3}$.
\end{enumerate}
\end{miniexercices}


\auteurs{
Arnaud Bodin, 
Niels Borne, 
Laura Desideri
}

\finchapitre
\end{document}


