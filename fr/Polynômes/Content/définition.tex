% !TeX root = ...
{
  base={
    authors={
      jlaurens,
    },
    title=Définition de polynôme,
    text={
Un \MDBDefine{polynôme}\MDBIndex{polynôme} à coefficients dans \(\MDBMeta{𝕂}\)
est une expression de la forme
\begin{gather*}
\MDBMeta{P}(\MDBMeta{X}) =
\MDBMeta{a}_{\MDBMeta{n}} \MDBMeta{X}^{\MDBMeta{n}}
+ \MDBMeta{a}_{\MDBMeta{n}-1} \MDBMeta{X}^{\MDBMeta{n}-1}
+ ⋯
+ a₂ \MDBMeta{X}²
+ \MDBMeta{a}₁ \MDBMeta{X}
+ a₀,
\end{gather*}
avec \(\MDBMeta{n}∈ \MDBMeta{ℕ}\) et \(\MDBMeta{a}₀\), \(\MDBMeta{a}₁\), ... \(\MDBMeta{a}_{\MDBMeta{n}} ∈ \MDBMeta{𝕂}\).

L'ensemble des polynômes est noté $\MDBMeta{𝕂}[\MDBMeta{X}]$.

\begin{itemize}
\item
Les $\MDBMeta{a}_{\MDBMeta{i}}$ sont appelés les \MDBDefine{coefficients}\MDBIndex{coefficient} du polynôme.

\item
Si tous les coefficients $\MDBMeta{a}_{\MDBMeta{i}}$ sont nuls, $\MDBMeta{P}$ est appelé le \MDBDefine{polynôme nul}, il est noté $0$.

\item 
On appelle le \MDBDefine{degré}\MDBIndex{degré} de $\MDBMeta{P}$ le plus grand entier $\MDBMeta{i}$ tel que $\MDBMeta{a}_{\MDBMeta{i}}≠0$ ;
on le note $\deg \MDBMeta{P}$. Pour le degré du polynôme nul on pose par convention $\deg(0)=-∞$.

\item
Un polynôme de la forme $\MDBMeta{P}=\MDBMeta{a}₀$ avec $\MDBMeta{a}₀∈\MDBMeta{𝕂}$ est appelé un \MDBDefine{polynôme constant}. Si $\MDBMeta{a}₀≠0$, son degré est $0$.
\end{itemize}
    },
    comment={Extrait du cours exo7},
  },
}
